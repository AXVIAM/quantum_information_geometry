

\documentclass[11pt]{article}
\usepackage{amsmath,amssymb,graphicx}
\usepackage[a4paper,margin=1in]{geometry}
\usepackage{authblk}
\usepackage{hyperref}

\title{A Bounce Cosmology from Quantum Information Gravity\\
\large Memory-Driven Curvature and the Structural Reversal of Spacetime}
\author{Christopher Patrick Booth Smolen}
\date{\today}

\begin{document}

\maketitle

\begin{abstract}
We present a bounce cosmology derived from Quantum Information Gravity (QIR), in which curvature arises from coherent memory rather than mass-energy. Unlike traditional bounce models that rely on quantum corrections, scalar fields, or brane collisions, the QIR bounce emerges naturally from the dynamics of informational structure. As the universe contracts, memory coherence saturates, producing a curvature floor that prevents singularity. Expansion follows as a reversal through the memory phase function, making the bounce a structural recursion. This model predicts non-singular early-universe behavior, entropy continuity, and potentially observable deviations from inflationary predictions. It unifies cosmic evolution as an expression of structural memory, rather than arbitrary initial conditions.
\end{abstract}

\section{Introduction}
Bounce cosmologies offer a path beyond the Big Bang singularity by proposing that the universe underwent a prior contraction followed by expansion. Yet most models invoke mechanisms that are either speculative or disconnected from physical structure. Quantum Information Gravity (QIR) introduces a new approach. In QIR, curvature is generated not solely by stress-energy, but by the density and coherence of memory encoded in structure. This allows the early universe to approach a fold, not a divergence, and to re-expand via harmonic reversal. This paper presents a full bounce cosmology built from QIR principles.

\section{Curvature from Memory: The QIR Framework}
QIR modifies the Einstein field equations by introducing a memory tensor \( I_{\mu\nu} \), sourced from the entropy-area ratio of systems:
\[
G_{\mu\nu} + \Lambda g_{\mu\nu} = \frac{8\pi G}{c^4} T_{\mu\nu} + \lambda I_{\mu\nu}
\]
where:
\[
I_{\mu\nu} = \kappa \cdot \partial_\mu \Phi \partial_\nu \Phi,\quad \Phi = \log\left(1 + \frac{S}{A}\right)
\]
This term encodes the structural coherence of the universe. When applied to cosmological metrics, it introduces a curvature limit driven by information saturation.

\section{The Memory Floor and Curvature Saturation}
QIR defines a minimal memory coherence:
\[
I(T) \geq \pi
\]
This implies that as the universe contracts, its curvature—sourced from increasing memory density—approaches but never exceeds a harmonic floor. Rather than diverging, the structure folds, reaching a saturated state where further compression is prohibited. This prevents the singularity that afflicts classical GR.

\section{Expansion as Harmonic Reversal}
The QIR entropy function follows a sinusoidal pattern:
\[
I(T) = \pi + (I_{\text{max}} - \pi) \cdot \sin^2\left(\frac{\omega t}{2} + \phi\right)
\]
This produces a natural reversal in the curvature source. Expansion is not a quantum bounce or scalar recoil—it is structural memory unfolding. Time is not broken at the bounce—it loops, preserving continuity.

\section{Comparison to Other Bounce Models}
Unlike loop quantum cosmology, the QIR bounce does not require quantization of space. Unlike ekpyrotic models, it requires no branes or extra dimensions. And unlike scalar field models, it requires no artificial inflation field. The bounce emerges directly from the physical principle that curvature tracks memory, and that memory cannot vanish or diverge.

\section{Predictions and Observables}
QIR bounce cosmology implies:
\begin{itemize}
  \item No singularity at \( t = 0 \)
  \item Time-symmetric entropy evolution
  \item Structure formation seeded by memory decoherence
  \item Potential deviations in the CMB acoustic peak phase alignment
  \item Recurring universes with entangled curvature echoes
\end{itemize}

These effects are testable in early-universe probes and structure surveys.

\section{Conclusion}
Quantum Information Gravity offers a complete and testable bounce cosmology. It replaces the Big Bang singularity with a memory fold, and expansion with a structural return. The universe does not begin from nothing—it remembers its way forward. In QIR, time is not broken—it is curved by what it has become.

\end{document}